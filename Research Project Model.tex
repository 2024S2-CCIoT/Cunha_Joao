%% abtex2-modelo-projeto-pesquisa.tex, v-1.9.7 laurocesar
%% Copyright 2012-2018 by abnTeX2 group at http://www.abntex.net.br/
%%
%% This work may be distributed and/or modified under the
%% conditions of the LaTeX Project Public License, either version 1.3
%% of this license or (at your option) any later version.
%% The latest version of this license is in
%%   http://www.latex-project.org/lppl.txt
%% and version 1.3 or later is part of all distributions of LaTeX
%% version 2005/12/01 or later.
%%
%% This work has the LPPL maintenance status `maintained'.
%%
%% The Current Maintainer of this work is the abnTeX2 team, led
%% by Lauro César Araujo. Further information are available on
%% http://www.abntex.net.br/
%%
%% This work consists of the files abntex2-modelo-projeto-pesquisa.tex
%% and abntex2-modelo-references.bib
%%

% ------------------------------------------------------------------------
% ------------------------------------------------------------------------
% abnTeX2: Modelo de Projeto de pesquisa em conformidade com
% ABNT NBR 15287:2011 Informação e documentação - Projeto de pesquisa -
% Apresentação
% ------------------------------------------------------------------------
% ------------------------------------------------------------------------

\documentclass[
	% -- opções da classe memoir --
	12pt,				% tamanho da fonte
	openright,			% capítulos começam em pág ímpar (insere página vazia caso preciso)
	twoside,			% para impressão em recto e verso. Oposto a oneside
	a4paper,			% tamanho do papel.
	% -- opções da classe abntex2 --
	%chapter=TITLE,		% títulos de capítulos convertidos em letras maiúsculas
	%section=TITLE,		% títulos de seções convertidos em letras maiúsculas
	%subsection=TITLE,	% títulos de subseções convertidos em letras maiúsculas
	%subsubsection=TITLE,% títulos de subsubseções convertidos em letras maiúsculas
	% -- opções do pacote babel --
	english,			% idioma adicional para hifenização
	french,				% idioma adicional para hifenização
	spanish,			% idioma adicional para hifenização
	brazil,				% o último idioma é o principal do documento
	]{abntex2}

% ---
% PACOTES
% ---

% ---
% Pacotes fundamentais
% ---
\usepackage{lmodern}			% Usa a fonte Latin Modern
\usepackage[T1]{fontenc}		% Selecao de codigos de fonte.
\usepackage[utf8]{inputenc}		% Codificacao do documento (conversão automática dos acentos)
\usepackage{indentfirst}		% Indenta o primeiro parágrafo de cada seção.
\usepackage{color}				% Controle das cores
\usepackage{graphicx}			% Inclusão de gráficos
\usepackage{microtype} 			% para melhorias de justificação
% ---

% ---
% Pacotes adicionais, usados apenas no âmbito do Modelo Canônico do abnteX2
% ---
\usepackage{lipsum}				% para geração de dummy text
% ---

% ---
% Pacotes de citações
% ---
\usepackage[brazilian,hyperpageref]{backref}	 % Paginas com as citações na bibl
\usepackage[alf]{abntex2cite}	% Citações padrão ABNT

% ---
% CONFIGURAÇÕES DE PACOTES
% ---

% ---
% Configurações do pacote backref
% Usado sem a opção hyperpageref de backref
\renewcommand{\backrefpagesname}{Citado na(s) página(s):~}
% Texto padrão antes do número das páginas
\renewcommand{\backref}{}
% Define os textos da citação
\renewcommand*{\backrefalt}[4]{
	\ifcase #1 %
		Nenhuma citação no texto.%
	\or
		Citado na página #2.%
	\else
		Citado #1 vezes nas páginas #2.%
	\fi}%
% ---

% ---
% Informações de dados para CAPA e FOLHA DE ROSTO
% ---
\titulo{Como os Serviços da AWS Podem Ajudar na Indústria}
\autor{João Victor Cunha}
\local{SENAI, Cloud Computing e IoT}
\data{2024}
\instituicao{%
  UniSenai
  \par
  Faculdade da Industria
  \par
  Engenharia de Software}
% O preambulo deve conter o tipo do trabalho, o objetivo,
% o nome da instituição e a área de concentração
% ---

% ---
% Configurações de aparência do PDF final

% alterando o aspecto da cor azul
\definecolor{blue}{RGB}{41,5,195}

% informações do PDF
\makeatletter
\hypersetup{
     	%pagebackref=true,
		pdftitle={\@title},
		pdfauthor={\@author},
    	pdfsubject={\imprimirpreambulo},
	    pdfcreator={LaTeX with abnTeX2},
		pdfkeywords={abnt}{latex}{abntex}{abntex2}{projeto de pesquisa},
		colorlinks=true,       		% false: boxed links; true: colored links
    	linkcolor=blue,          	% color of internal links
    	citecolor=blue,        		% color of links to bibliography
    	filecolor=magenta,      		% color of file links
		urlcolor=blue,
		bookmarksdepth=4
}
\makeatother
% ---

% ---
% Espaçamentos entre linhas e parágrafos
% ---

% O tamanho do parágrafo é dado por:
\setlength{\parindent}{1.3cm}

% Controle do espaçamento entre um parágrafo e outro:
\setlength{\parskip}{0.2cm}  % tente também \onelineskip

% ---
% compila o indice
% ---
\makeindex
% ---

% ----
% Início do documento
% ----
\begin{document}

% Seleciona o idioma do documento (conforme pacotes do babel)
%\selectlanguage{english}
\selectlanguage{brazil}

% Retira espaço extra obsoleto entre as frases.
\frenchspacing

% ----------------------------------------------------------
% ELEMENTOS PRÉ-TEXTUAIS
% ----------------------------------------------------------
% \pretextual

% ---
% Capa
% ---
\imprimircapa
% ---




% ---
% inserir o sumario
% ---
\pdfbookmark[0]{\contentsname}{toc}
\tableofcontents*
\cleardoublepage
% ---


% ----------------------------------------------------------
% ELEMENTOS TEXTUAIS
% ----------------------------------------------------------
\textual

% ----------------------------------------------------------
% Introdução
% ----------------------------------------------------------
\chapter*[Introdução]{Introdução}
\addcontentsline{toc}{chapter}{Introdução}

Nos últimos anos, a transformação digital tornou-se um imperativo para diversas indústrias ao redor do mundo. O avanço das tecnologias da informação e comunicação, aliado ao uso de soluções em nuvem, tem possibilitado mudanças significativas em processos produtivos e operacionais. Nesse cenário, a Amazon Web Services (AWS) se destaca como uma das principais provedoras de infraestrutura em nuvem, oferecendo uma ampla gama de serviços que atendem às necessidades de diferentes setores industriais.

A migração para a nuvem, assim como a implementação de tecnologias como Internet das Coisas (IoT), Machine Learning e análise de dados, está permitindo às empresas não apenas melhorar a eficiência e reduzir custos, mas também explorar novas oportunidades de inovação. A flexibilidade, escalabilidade e segurança fornecidas pela AWS tornam-se essenciais para enfrentar os desafios de um mercado cada vez mais competitivo e globalizado \cite{armbrust2010}.

A Internet das Coisas Industrial (IIoT) permite a conexão em tempo real de fábricas e sistemas industriais, aumentando a eficiência dos processos produtivos e da logística \cite{gilchrist2016}. Além disso, o uso de Machine Learning em sistemas industriais está transformando a forma como as empresas otimizam seus processos e tomam decisões de forma autônoma, gerando um impacto significativo na produtividade \cite{zhong2017}. A análise de grandes volumes de dados também tem desempenhado um papel crucial na tomada de decisões baseada em dados, ajudando empresas a melhorar sua eficiência operacional \cite{manyika2011}.

Este trabalho tem como objetivo investigar de que forma os serviços da AWS podem ser aplicados em setores industriais como manufatura, energia e logística, demonstrando como tais tecnologias podem contribuir para o aumento da eficiência, a redução de custos operacionais e a inovação de processos. Além disso, o estudo busca explorar os principais benefícios e desafios da adoção dessas soluções em nuvem, apresentando casos de sucesso de empresas que implementaram com êxito tecnologias da AWS em seus processos industriais.

Por meio de uma análise teórica e estudos de casos, espera-se que este trabalho forneça uma visão abrangente sobre o papel fundamental que a AWS pode desempenhar na modernização da indústria, alinhando-se às exigências da era digital e garantindo a competitividade das empresas no cenário atual.

\phantompart

\printindex


\end{document}
